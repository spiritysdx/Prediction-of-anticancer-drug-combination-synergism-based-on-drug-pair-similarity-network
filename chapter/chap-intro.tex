% !Mode:: "TeX:UTF-8"
\chapter{绪论}
\label{chap:intro}

\section{课题背景}

与单一药物治疗相比,通过药物组合治疗复杂疾病是一种成熟方法\supercite{1},如癌症\supercite{2}。癌症是一种严重危害人类健康的疾病,因此寻找有效的治疗方法至关重要。使用药物组合而不是单一药物治疗,能够降低每种药物的剂量,减轻毒副作用\supercite{4},并有效克服耐药性\supercite{5},提高疗效。因此,了解药物组合之间的协同作用机制,制定精准的预测模型,能够快速、便捷地寻找针对特定癌症类型的药物组合,对于提高癌症治疗效果具有重要意义\supercite{3}。

在药物组合研究领域,存在着许多难题。其中最大的难题之一是随着药物数量的增加,药物组合可能的空间大小会飞速增长。此外,药物组合不仅会存在协同作用,也可能存在拮抗或者相加作用\supercite{6}\supercite{7}。药物组合治疗可能是不利的,甚至导致癌症患者的无进展生存期缩短\supercite{26}。尽管人们对药物协同作用的机制有所探索,但目前大多数协同作用的药物组合仍是根据临床经验提出的。早期,这种研究方法需要通过临床经验总结提出有效的药物组合,这是时间、劳动力和成本密集型的方法,还可能会给病人带来不必要甚至有害的治疗\supercite{8}。因此,高通量筛选(HTS)成为了另一种在不伤害病人的情况下识别协同药物组合的方法。这种方法可以在合理的时间内以较低的成本产生大量的测量结果\supercite{9}。在这些筛选中,不同浓度的两种药物被应用于一个癌症细胞系。尽管癌症细胞系的研究在生物医学研究中很重要,但它们准确代表体内状态的能力经常受到质疑。原因是,即使原始肿瘤和衍生的癌症细胞系之间有很高的基因组相关性,但仍然远远不够完美\supercite{27}。此外,使用HTS测试完整的药物组合空间仍然是不可行的,因为它需要大量的资源进行基础设施的建设\supercite{10}。因此,计算方法提供了有效探索大型协同空间的可能。通过利用现有的HTS协同效应数据,可以生成性能较高的预测模型,为体内外研究提供指导方向。

\section{课题国内外现状}

近年来,人工智能技术的蓬勃发展促进了计算方法的发展。基于机器学习(ML)和深度学习(DL)的方法可以模拟复杂的非线性过程,弥补了早期研究中基于显性模型的计算方法只适用于特定目标、途径、疾病或细胞的缺点\supercite{11}。这些方法的经典模式是首先构造细胞系和药物的特征,然后使用ML或DL模型进行预测。在2020年以前,许多研究采用了各种ML模型来预测抗癌药物组合的协同作用,包括随机森林、逻辑回归、XGBoost和极度随机树等网络工具\supercite{12}。这些ML模型为预测药物组合提供了便捷的服务。2015 年张乃千等人基于GDSC和CCLE数据集构建了整合的细胞系-药物双层网络模型,提出局部线性方法预测抗癌药物对于细胞系的敏感性,较以往结果有了很大的改进\supercite{15}。随着时间的推移,研究人员开始使用DL模型\supercite{22}。目前较好的DL方法是DeepSynergy\supercite{13}和PRODeepSyn\supercite{28}。DeepSynergy使用化学和基因组信息作为输入信息,通过凸半螺旋层模拟药物协同作用的规律性。它是回归模型,因为把任务作为分类问题可能会过度简化实际情况\supercite{14}。PRODeepSyn则将蛋白质相互作用(PPI)网络数据与组学数据相结合,生成细胞系的低维稠密嵌入。该方法包括一个图卷积网络(GCN),该网络提取生物网络的信息特征,并使用具有批标准化机制的深度神经网络(DNN)预测协同得分。这些DL方法的Pearson相关系数(PCC)普遍可达0.70以上,其他ML方法则大多在0.66以下\supercite{28}。

\section{本文的主要工作及内容安排}

探索并开发高效的生物序列分析计算模型以充分挖掘相关数据中所蕴涵的规律与知识,有助于理解和最终揭示生命的本质。目前,对生物数据信息的挖掘和分析已经成为当代生命科学甚至自然科学的重要前沿研究领域,其有效的数据挖掘方法已经成为替代部分昂贵生物医学实验的有效途径之一。

2015年,张乃千等研究团队发现了一个现象,即"基因组信息相似的细胞系和分子结构相似的药物对应的药物反应谱更为相似"。他们利用了这个现象结合局部线性方法,构建了一个细胞系-药物双层网络模型,利用邻居细胞系和邻居药物的敏感性信息,取得了良好的预测效果\supercite{15}。受以上工作的启发,本文基于假设“相似的药物组合对固定细胞系具有相似的协同作用”,构建基于药物组合的相似性网络模型,去预测新药物组合对固定细胞系的协同作用。为了解决这个问题,本文采用了四种方法求解模型的超参数,最后采用了其中效果最优的精细调参的方法求解模型,在实验中得到的PCC普遍高于0.675,相较于许多现有的机器学习方法,本模型取得了显著进步。此外,本文使用了两个数据集进行实验,均获得了良好的效果,说明该模型具有很好的泛化能力,同时该模型复杂度较ML模型更为简单,时间成本更低。

第1章介绍了课题的背景、研究目的及意义,并探讨国内外相关研究的现状。首先概述了药物组合研究的起源和发展,介绍了相关ML方法和DL方法的主要应用领域。其次阐述了用于预测抗癌药物组合的协同作用的相关方法和模型。最后,介绍了本文的主要工作内容以及本论文的结构安排。

第2章介绍了验证假设和构建药物组合相似性网络模型所使用的两个数据集。作三维散点图论证了对原始数据进行Z-Score行标准化数据预处理的必要性。

第3章首先介绍了药物组合的相似性计算方法,其次,由文献\supercite{15}的启发,提出了相似性假设,验证了使用Tanimoto系数的阈值来划分高相似药物组合和低相似药物组合的可行性。最后,本文构建了基于药物组合相似性网络模型,并提出采用四种方法求解模型的超参数。

第4章介绍了DCSN模型求解的4种方法,并对它们的求解结果和优势进行了分析与比较。针对模型的求解结果,检验了DCSN模型在两个数据集上分析拟合效果是否是显著的,检验了预测结果与实际测试值之间是否存在线性相关性。本章还对模型超参数对预测结果的影响进行了分析,并探讨了一些极端预测值出现的原因。为了比较相关模型,选择了文献\cite{13}和文献\cite{18}中所记录的与各数据集相对应的模型进行对比。