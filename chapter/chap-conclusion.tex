% !Mode:: "TeX:UTF-8"
\begin{conclusion}
\label{chap:conclusion}

本文聚焦抗癌药物组合协同作用预测问题,构建了形式简单、直观易懂的药物组合相似性网络模型DCSN,并将其应用于两大抗癌药物组合高通量筛选数据集,O'Neil 数据集和 NCI-ALMANAC 数据集,得到的具体结论如下: 

(1) 提出了药物组合相似性假设:分子指纹相似的药物组合对于固定细胞系具有相似的协同作用。该假设反映了药物组合分子指纹之间的相似度与其协同得分之间的关系,并且通过如下实验现象得到了验证:高相似药物组合团的协同得分相关性较强,低相似药物组合团的协同得分相关性较弱。 

(2) DCSN在O'Neil 数据集和 NCI-ALMANAC 数据集上都取得了较为理想的预测结果,预测协同得分与观测协同得分的PCC分别为0.675和0.684,优于部分经典的机器学习模型,但低于深度学习模型DNN。特别值得一提的是,DCSN仅涉及两个参数,分别是邻居数和衰减率,模型的计算成本极低。结果表明:DCSN可以作为新药物组合高通量虚拟筛选的可选择工具。 

尽管本文的研究工作取得了一些成果,但仍有改进的空间。后续进一步研究的问题如下:

(1) 在相似性分析中,除了本文采用的 Tanimoto 系数和 Spearman 相关系数,还可以考虑使用其他距离度量方法,以提高相似性分析的准确度。

(2) 可以将药物靶向信息、联合用药信息、细胞系的基因表达信息、基因组突变信息和拷贝数变异信息整合到模型中,提高药物组合协同作用预测的精度。

(3) 可以通过设定邻居数阈值来优化筛选和预测药物协同作用得分的方法。这样,即使某些药物没有高相似邻居,仍能进行筛选和预测,避免某种药物具有过多高相似邻居的情况,确保筛选方法具有实际应用意义。

(4) 基于本文相同的理念,未来可以考虑验证“相似的细胞系对于固定药物组 合具有相似的协同作用”这一假设,进而构建细胞系相似性网络模型,实现新细胞系的抗癌药物组合协同作用预测。

\end{conclusion}